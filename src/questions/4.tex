{[10 marks]} The {\it diameter} of a tree is the length of the longest simple path
between nodes of the tree. Design an efficient divide and conquer algorithm to compute the diameter of a rooted binary tree. Your algorithm may use the following operations on trees that each have time complexity $\Theta(1)$.
\begin{itemize}
    \item \textsc{Root}($T$) --- Returns the root node of the tree $T$.
    \item \textsc{LeftChild}($T$, $v$) --- Returns the node that is the left child of $v$ in $T$; or $\emptyset$ if $v$ does not have a left child.
    \item \textsc{RightChild}($T$, $v$) --- Returns the node that is the right child of $v$ in $T$; or $\emptyset$ if $v$ does not have a right child.
\end{itemize}

Provide a pseudocode and a $\Theta$ bound in the runtime analysis. A proof of correctness is always required.
\\

The implementation of the $\textsc{DiameterOfBinaryTree}$ algorithm is,
\begin{minted}[escapeinside=??, fontfamily=lmr]{text}
?\sffamily{\textbf{ALGORITHM}}? ?\textit{SelectSort}?(?\(A\left[0..n - 1\right]\)?)?\tikzmark[qfourL1]{qfourL1}?
?/\!/?? Selection sort algorithm.??\tikzmark[qfourL2]{qfourL2}?
?\textbf{for}? ?\(i \leftarrow 0\)? ?\textbf{to}? ?\(n - 1\)? ?\textbf{do}??\tikzmark{qfourL3}?
    ?\(SmallSub \leftarrow i\)?
    ?\textbf{for}? ?\(j \leftarrow i + 1\)? ?\textbf{to}? ?\(n - 1\)? ?\textbf{do}??\tikzmark{qfourL5}?
        ?\textbf{if}? ?\(A[j] < A[SmallSub]\)? ?\textbf{then}??\tikzmark{qfourL6}?
            ?\(SmallSub \leftarrow j\)??\tikzmark{qfourL7}?

    ?\(temp \leftarrow A[i]\)?
    ?\(A[SmallSub] \leftarrow temp\)??\tikzmark{qfourL10}?

?\textbf{return}? ?\(A\)?
\end{minted}
\begin{tikzpicture}[remember picture,overlay]
\coordinate (qfourBASEX) at (7, 0);
\draw ({pic cs:qfourL1}-|qfourBASEX) coordinate (qfourA);
\draw ({pic cs:qfourL2}-|qfourBASEX) coordinate (qfourB);
\tikzmath{coordinate \qfourDVEC;
    \qfourDVEC = (qfourA)-(qfourB);
    \qfourD = (\qfourDVECy);
    \qfourD = \convertto{cm}{\qfourD pt};
}
\draw (qfourA) ++ (0, {0.75cm*\qfourD}) [decorate, decoration = {calligraphic brace, amplitude=5pt}, line width=1.25pt]
++ (0, {-5*\qfourD}) --++ (0, {-1*\qfourD}) node [black,midway,xshift=0.2cm,anchor=west] {\(O\left( 1 \right)\)};
\draw (qfourA) ++ (1.5, {0.75cm*\qfourD}) [decorate, decoration = {calligraphic brace, amplitude=5pt}, line width=1.25pt]
++ (0, {-4*\qfourD}) --++ (0, {-3*\qfourD}) node [black,midway,xshift=0.2cm,anchor=west] {\(O\left( n \right)\)};
\draw (qfourA) ++ (3.0, {0.75cm*\qfourD}) [decorate, decoration = {calligraphic brace, amplitude=5pt}, line width=1.25pt]
++ (0, {-2*\qfourD}) --++ (0, {-8*\qfourD}) node [black,midway,xshift=0.2cm,anchor=west] {\(O\left( n^2 \right)\)};
\end{tikzpicture}
\leavevmode\newline
The algorithm is essentially a modified max height algorithm. The height at a particular level is
the height of the left and right sub-branches plus 1, which is the result returned from the function.
These results can also be used to find the diameter which passes through the current node, simply by
adding the depths of the left and right branches. This is because the total number of nodes in the path
will be $\text{height of left} + \text{height of right} + 1$, which means the number of edges (diameter) is
$\text{height of left} + \text{height of right}$. Lastly, the diameter of the tree is modified to be
the maximum of all the diameters which pass through each node. \\

Since the algorithm involves a depth-first traversal of the tree with constant time operations at each
level, we can conclude that the algorithm $\in \Theta\left(n\right)$, where $n$ is the number of nodes
in the tree.