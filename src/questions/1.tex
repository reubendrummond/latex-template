\begin{enumerate}
    \item {[9 marks]} Prove or disprove each of the following statements.
          \begin{enumerate}
              \item if $\log (f(n)) \in \Omega(\log(g(n)))$ then $f(n) \in \Omega(g(n))$. \\

                    Consider functions for $k \in \N$,

                    \begin{align*}
                        f(n) & = n^{k}   \\
                        g(n) & = n^{k+1} \\
                    \end{align*}

                    We have that,

                    \begin{align*}
                        \log{f(n)} & = kn                   \\
                        \log{g(n)} & = \left(k+ 1 \right) n
                    \end{align*}

                    It is true that $\log{f(n)} \in \Omega\left(g \left( n \right) \right)$ since,

                    \begin{align}
                        \exists c, n_0 > 0: \forall n > n_0, c \log{g \left( n \right)} \le \log{f \left( n \right)} \label{eg:Omega}
                    \end{align}

                    And we can select a $c \forall n_0$ which satisfies the definition,

                    \begin{align*}
                        c \log{g \left( n \right)} & \le \log{f \left(n \right)}                                    \\
                        c kn                       & \le \left(k + 1 \right) n                                      \\
                        c                          & \le \frac{\left(k + 1 \right)}{k} \quad \text{(since $n > 0$)} \\
                    \end{align*}

                    Assume that $f \left( n \right) \in \Omega g \left( n \right)$. By \ref{eq:Omega}, we can see that,

                    \begin{align*}
                        c g \left( n \right) & \le f \left(n \right)         \\
                        c n^{k + 1}          & \le n^k                       \\
                        c                    & \le \frac{n^{k + 1}}{n^k} = n
                    \end{align*}

                    Clearly, we cannot select a finite value of $c$ for which the condition for \ref{eg:Omega} hold, which is
                    a contradiction to the claim that $f \left( n \right) \in \Omega g \left( n \right)$. \\

                    Since we have found a counter example we can disprove the original statement. \\

              \item If $f(n) \in O(h(n))$ and $g(n) \in O(h(n))$ then $\dfrac{f(n)}{g(n)} \in O(1)$.

                    Consider the functions,

                    \begin{align*}
                        f\left(n\right) & = n   \\
                        g\left(n\right) & = 1   \\
                        h\left(n\right) & = n^2 \\
                    \end{align*}

                    Clearly, $f(n), g(n) \in h(n)$ since,

                    \begin{align*}
                        \exists c_1, n_0 : \forall n > n_0, & f\left(n\right) \le c_1 h\left(n\right) \\
                        \exists c_2, n_0 : \forall n > n_0, & g\left(n\right) \le c_2 h\left(n\right) \\
                    \end{align*}

                    Now consider,
                    \begin{align*}
                        \frac{f\left(n\right)}{g\left(n\right)} & = \frac{n}{1} = n \notin O{1}
                    \end{align*}

                    We can therefore disprove the original statement by counter example. \\

              \item If $f(n) \in o(g(n))$ then $\log(f(n)) \in o(\log(g(n)))$.

                    % If the statement is true, we must have,

                    % \begin{align}
                    %     \forall c > 0, \exists n_0 > 0: \forall n > n_0, f \left( n \right) \le c g \left( n \right) \label{eg:little_o}
                    % \end{align}

                    % We have that,

                    % \begin{align*}
                    %     f \left( n \right)         & \le c \log{g \left( n \right)}         \\
                    %     \log{ f \left( n \right) } & \le \log{ c \log{g \left( n \right)}}  \\
                    %     \log{ f \left( n \right) } & \le \log{c} + \log{g \left( n \right)} \\
                    % \end{align*}

                    Consider functions such that $f\left(n\right) \in o\left(g\left(n\right)\right)$,

                    \begin{align*}
                        f\left(n\right) & = n   \\
                        g\left(n\right) & = n^2 \\
                    \end{align*}

                    If $\log{f \left( n \right)} \in o\left( \log{g \left( n \right)} \right)$ we must
                    have $\forall c > 0$,

                    \begin{align*}
                        \log{f \left( n \right)} & \le c \log{g \left( n \right)}     \\
                        \log{n}                  & \le c \log{n^2}                    \\
                        \log{n}                  & \le c 2 \log{n}                    \\
                        1                        & \le c 2 \quad \text{since $n > 0$} \\
                        \frac{1}{2}              & \le c                              \\
                        c \ge \frac{1}{2}
                    \end{align*}

                    Since $c \ge \frac{1}{2}$, we conclude that $\log{f \left( n \right)} \notin o\left( \log{g \left( n \right)} \right)$.
                    Therefore, we have disproven the original statement by counter example.

          \end{enumerate}
    \item {[6 marks]} Analyze the following pseudocode and give a tight ($\Theta$) bound on the running time
          as a function of $n$. You can assume that all individual instructions
          (including logarithm) are elementary, i.e., take constant time. Show your work.

          \verb|     |$l$ \verb| := | 0;\\
          \verb|     for |$i=n+1$ \verb| to | $ n^2 $ \verb| do |\\
          \verb|       for |$j=1$ \verb| to | $\left\lceil \log{i} \right\rceil$ \verb|  do |\\
          \verb|          |$l $\verb| := | $l+1$\\
          \verb|       od  |\\
          \verb|     od.|\\

          There are two loops which can be expressed as a nested sum. Note that the execution time of operations
          are denoted as $c_1$ through to $c_4$ in order. Expressed as a sum, we have that the running time is,

          \begin{align*}
              \text{running time} & = c_1 + \sum_{i=n+1}^{n^2} \left( c_2 + \sum_{j=0}^{\log{i}} \left( c_3 + c_4 \right) \right) \\
                                  & = c_1 + \sum_{i=n+1}^{n^2} \left( c_2 + \left( c_3 + c_4 \right) \log{i} \right)              \\
          \end{align*}

          We can say that,

          \begin{align*}
              \sum_{i=n+1}^{n^2} \left( c_2 + \left( c_3 + c_4 \right) \log{i} \right) \le \int_{x}^{x^2} \left( c_2 + \left( c_3 + c_4 \right) \log{x} \right) dx
          \end{align*}

          Since $\log{x}$ is monotonically increasing and the sum can be viewed as a Riemann sum of width $1$. \\

          Noting that,

          \begin{align*}
              \int \log_b{x} dx = \frac{1}{\ln{b}} \left( x\ln{x} - x \right) + c
          \end{align*}

          we have,

          \begin{align*}
              \text{running time} & \le c_1 + \int_{n}^{n^2} \left( c_2 + \left( c_3 + c_4 \right) \log_b{x} \right) dx                                                                          \\
                                  & \le c_1 + \left[ c_2 x + \left( c_3 + c_4 \right) \frac{1}{\ln{b}} \left( x\ln{x} - x \right) \right]_n^{n^2}                                                \\
                                  & =  c_1 + c_2 \left( n^2 - n \right) + \left( c_3 + c_4 \right) \frac{1}{\ln{b}} \left[ \left( n^2\ln{n^2} - n^2 \right) - \left( n\ln{n} - n \right) \right]
          \end{align*}

          Clearly, the highest order term is proportional to $n^2 \log{n}$. Therefore, we can conclude the tight
          bound is $\Theta\left(n^2 \log{n}\right)$.

\end{enumerate}